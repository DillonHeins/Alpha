\documentclass{article}

\usepackage{lipsum}
\usepackage[margin=2.5cm, left=2cm, includefoot]{geometry}

% Graphics preamble
\usepackage{graphicx} % Allows you to import images
\usepackage{float} % Allows for control of float positions
%

% Header and footer
\usepackage{fancyhdr}
\pagestyle{fancy}

\fancyhead{}
\fancyfoot{}
\fancyfoot[R]{\thepage}
\renewcommand{\headrulewidth}{0pt}
\renewcommand{\footrulewidth}{0pt}
%

\begin{document}

\begin{titlepage}
	\begin{center}
		\line(1,0){300}\\
		[6mm]
		\huge{\bfseries Testing - Alpha Integration}\\
		[2mm]
		\line(1,0){200}\\
		[15mm]
		\textsc{\large Members}\\
		[7.5mm]
		\textsc{
                \large14011396	Grobler	A (Arno) \\
                14103207	Mogase	L (Lethabo)\\
                12017800	Maree	AA (Armand)\\
                12059138	Saaiman	C (Christiaan)\\
                13027205	Smallwood	DW (Duncan)\\
                13040686	Nyuswa	ML (Maluleki)\\
                14035538	Heins	D (Dillon)\\
                14395283	Alberts	JD (Josef)\\
                12031748	Khumalo	S (Sandile)\\
                13278012	Muranga	KJ (Kudzai)\\
                14222583	Molefe	KP (Keletso)\\
            }\\
		[8cm]
	\end{center}
	
	\begin{flushright}
		\textsc{\large Lorem\\
		Ipsum\\
		20 April 2016\\}
	\end{flushright}
\end{titlepage}

% Table of contents
\tableofcontents
\thispagestyle{empty}
\cleardoublepage
%

% Main body
\pagenumbering{arabic}
\setcounter{page}{1}
%

\section{Introduction}\label{sec:intro}
    This project is a Research Support System. Its main purpose is to keep track of many aspects that are regularly needed by researchers and a easy way for research heads or HOD of the respective facility to be able to see progress that has been made by the researchers in their team.
    
    Scope for this project is only the Computer Science department and its respective lecturers and leaders.
    General uses of this system should include keeping track of:
    \begin{itemize}
        \item Research Papers
        \item Reports, of which has different types
        \item Research Groups
        \item Running Costs
        \item Historical publications
        \item List of Authors
        \item List of Users
        \item Units 
    
    \end{itemize}
    \subsection{Testing}
        Software testing plays an important part of the software development pipeline and is necessary for creating maintainable and scalable programmes. It has many different objectives and goals but ultimately it comes down to:
        \begin{itemize}
            \item Finding defects in the program that was developed and possibly finding solutions to those problems
            \item To ensure the level of quality has been maintained throughout the program
            \item Find potential security flaws and vulnerabilities
            \item Ensuring usability
            \item Ensuring the program has lived up to the system requirements set out by the clients.
        \end{itemize}
    Testing ensures that the project, which should be in its final stages of development, is ready to be rolled out to the clients and that it meets the requests made by the clients. Thus certain fields of the program could be addressed. These fields are:
        \begin{itemize}
            \item Flexibility
            \item Maintainability
            \item Scalability
            \item Performance requirements
            \item Reliability
            \item Security
            \item Auditability
            \item Testability
            \item Usability
            \item Integrability
            \item Deployability
        \end{itemize}

\newpage
\section{Front-end Interfaces: Web and Android Development}
\subsection{Web Development}
            \subsubsection{Flexibility}
            \subsubsection{Maintainability}
            \subsubsection{Scalability}
            \subsubsection{Performance requirements}
            \subsubsection{Reliability}
            \subsubsection{Security}
            \subsubsection{Auditability}
            \subsubsection{Testability}
            \subsubsection{Usability}
            \subsubsection{Integrability}
            \subsubsection{Deployability}
            \subsubsection{Correctly implemented functionality}
            \subsubsection{Short-comings of the implemented functionality}
            \subsubsection{Missing functionality}
\subsection{Android}
            \subsubsection{Flexibility}
            \subsubsection{Maintainability}
            \subsubsection{Scalability}
            \subsubsection{Performance requirements}
            \subsubsection{Reliability}
            \subsubsection{Security}
            \subsubsection{Auditability}
            \subsubsection{Testability}
            \subsubsection{Usability}
            \subsubsection{Integrability}
            \subsubsection{Deployability}
            \subsubsection{Correctly implemented functionality}
            \subsubsection{Short-comings of the implemented functionality}
            \subsubsection{Missing functionality}
\newpage
\section{Body}
\lipsum[1]

\subsection{Subsection}
\subsubsection{Subsubsection}

\end{document}

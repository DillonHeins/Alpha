\documentclass{article}

\usepackage{lipsum}
\usepackage[margin=2.5cm, left=2cm, includefoot]{geometry}

% Graphics preamble
\usepackage{graphicx} % Allows you to import images
\usepackage{float} % Allows for control of float positions
%

% Header and footer
\usepackage{fancyhdr}
\pagestyle{fancy}

\fancyhead{}
\fancyfoot{}
\fancyfoot[R]{\thepage}
\renewcommand{\headrulewidth}{0pt}
\renewcommand{\footrulewidth}{0pt}
%

\begin{document}

\begin{titlepage}
	\begin{center}
		\line(1,0){300}\\
		[6mm]
		\huge{\bfseries Testing - Alpha Integration}\\
		[2mm]
		\line(1,0){200}\\
		[15mm]
		\textsc{\large Members}\\
		[7.5mm]
		\textsc{
                \large14011396	Grobler	A (Arno) \\
                14103207	Mogase	L (Lethabo)\\
                12017800	Maree	AA (Armand)\\
                12059138	Saaiman	C (Christiaan)\\
                13027205	Smallwood	DW (Duncan)\\
                13040686	Nyuswa	ML (Maluleki)\\
                14035538	Heins	D (Dillon)\\
                14395283	Alberts	JD (Josef)\\
                12031748	Khumalo	S (Sandile)\\
                13278012	Muranga	KJ (Kudzai)\\
                14222583	Molefe	KP (Keletso)\\
            }
		[8cm]
	\end{center}
	
	\begin{flushright}
		\textsc{\large Lorem\\
		Ipsum\\
		20 April 2016\\}
	\end{flushright}
\end{titlepage}

% Table of contents
\tableofcontents
\thispagestyle{empty}
\cleardoublepage
%

% Main body
\pagenumbering{arabic}
\setcounter{page}{1}
%

\section{Introduction}\label{sec:intro}
    This project is a Research Support System. Its main purpose is to keep track of many aspects that are regularly needed by researchers and a easy way for research heads or HOD of the respective facility to be able to see progress that has been made by the researchers in their team.
    
    Scope for this project is only the Computer Science department and its respective lecturers and leaders.
    General uses of this system should include keeping track of:
    \begin{itemize}
        \item Research Papers
        \item Reports, of which has different types
        \item Research Groups
        \item Running Costs
        \item Historical publications
        \item List of Authors
        \item List of Users
        \item Units 
    
    \end{itemize}
    \subsection{Testing}
        Software testing plays an important part of the software development pipeline and is necessary for creating maintainable and scalable programmes. It has many different objectives and goals but ultimately it comes down to:
        \begin{itemize}
            \item Finding defects in the program that was developed and possibly finding solutions to those problems
            \item To ensure the level of quality has been maintained throughout the program
            \item Find potential security flaws and vulnerabilities
            \item Ensuring usability
            \item Ensuring the program has lived up to the system requirements set out by the clients.
        \end{itemize}
    Testing ensures that the project, which should be in its final stages of development, is ready to be rolled out to the clients and that it meets the requests made by the clients. Thus certain fields of the program could be addressed. These fields are:
        \begin{itemize}
            \item Flexibility
            \item Maintainability
            \item Scalability
            \item Performance requirements
            \item Reliability
            \item Security
            \item Auditability
            \item Testability
            \item Usability
            \item Integrability
            \item Deployability
        \end{itemize}
\section{Functional Testing}
	\subsection{Reporting}
		\subsubsection{Implementation}
		\subsubsection{Short-comings}
		\subsubsection{Missing}
		
	\subsection{Notifications}
		\subsubsection{Implementation}
		Unit tests were performed for all the functionality within the notification module.Dependency injection was used through the spring framework and everything worked.J unit was used accordingly and all the test cases that were written passed.
		\subsubsection{Short-comings}
		There are no short comings, all functionality specified by the specification were implemented.
		\subsubsection{Missing}
		
	\subsection{People}
		\subsubsection{Implementation}
		\subsubsection{Short-comings}
		\subsubsection{Missing}
		
	\subsection{Publications}
		\subsubsection{Implementation}
			The integration functionality to get a publication works, it takes in a json object request and returns a response with is a json object.if the publication does not exist the is a catch an exception or if the request in not valid it will catch an exception. for that they use a rest layer. This is the only functionality implemented by the integration team for publication.
		\subsubsection{Short-comings}
		There were no short commings in the get pulbication function
		\subsubsection{Missing}
		There is no other functionality to modify a publication such as to change publication state ,add publication type, modify publication type and also to getPublicationForPerson,getPublicationForGroup and calculate Accreditation points for both group and person.
		
	\subsection{Integration/Interface}
		\subsubsection{Implementation}
		\subsubsection{Short-comings}
		\subsubsection{Missing}

\section{Architectural Compliance}
	\subsection{Reporting}
		\subsubsection{Software Architecture Adhered}
		\subsubsection{Software Architecture Partially Adhered}
		\subsubsection{Software Architecture Not Adhered}		
		
	\subsection{Notifications}
		\subsubsection{Software Architecture Adhered}
		\subsubsection{Software Architecture Partially Adhered}
		\subsubsection{Software Architecture Not Adhered}		
		
	\subsection{People}
		\subsubsection{Software Architecture Adhered}
		\subsubsection{Software Architecture Partially Adhered}
		\subsubsection{Software Architecture Not Adhered}		
		
	\subsection{Publications}
		\subsubsection{Software Architecture Adhered}
		\subsubsection{Software Architecture Partially Adhered}
		\subsubsection{Software Architecture Not Adhered}		
		
	\subsection{Integration/Interface}
		\subsubsection{Software Architecture Adhered}
		\subsubsection{Software Architecture Partially Adhered}
		\subsubsection{Software Architecture Not Adhered}		
		
\section{Publication Architecture Compliance}
    \subsection{Flexibility}
    The integration code complies to this requirement because it has the ability to deploy different versions of the system with as little down-time as possible. This is partly due to the decoupled nature of the integration and publication sections. The code also makes use of dependency injection for contract based software development.
    
    \subsection{Maintainability}
    The code has very little to no documentation, which makes it very difficult for developers that did not on the system initially to be able to understand the code.
    
    \subsection{Scalability}
    The system is able to store a large amount of the department's publication meta data on the database without any difficulty.
    
    \subsection{Performance Requirements}
    All the the services of Publication operate within the required time of 0.2 seconds.
    
    \subsection{Auditability}
    The system does not log any of the requests, responses and exceptions of the publication services.
    
    
    \subsection{Reliability}
    The service requests are fully implemented and can be fully executed for each service contract for the publication section. It also supports commit and rollback functionality for the database.
    
    \subsection{Security}
    The integration code does not check the authorisation of the user for the services that only should be used by specific users. For example, the addPublicationType service should only allow administrators to use it. This is not implemented.
    
    \subsection{Testability}
    Junit is used for out-of-container testing to test the publication section. Embedded container unit testing is not used, however, this is listed as a nice-to-have.

\newpage
\section{Front-end Interfaces: Web and Android Development}
        Front end developers were tasked to use information from the back-end, such as database objects and requests and display it in a way that is user friendly to the user and intuitive for the designer. The interface also needs to provide the user easy tools to make requests to the back-end without the user having to worry about actual functionality of the system.
\subsection{Web Development}
        \subsubsection{Flexibility}
        \subsubsection{Maintainability}
        \subsubsection{Scalability}
        \subsubsection{Performance requirements}
        \subsubsection{Reliability}
        \subsubsection{Security}
        \subsubsection{Auditability}
        \subsubsection{Testability}
        \subsubsection{Usability}
        \subsubsection{Integrability}
        \subsubsection{Deployability}
        \subsubsection{Correctly implemented functionality}
        \subsubsection{Short-comings of the implemented functionality}
        \subsubsection{Missing functionality}
\subsection{Android}
        \subsubsection{Flexibility}
        \subsubsection{Maintainability}
        \subsubsection{Scalability}
        \subsubsection{Performance requirements}
        \subsubsection{Reliability}
        \subsubsection{Security}
        \subsubsection{Auditability}
        \subsubsection{Testability}
        \subsubsection{Usability}
        \subsubsection{Integrability}
        \subsubsection{Deployability}
        \subsubsection{Correctly implemented functionality}
        \subsubsection{Short-comings of the implemented functionality}
        \subsubsection{Missing functionality}
\newpage
\section{Body}
\lipsum[1]


\end{document}